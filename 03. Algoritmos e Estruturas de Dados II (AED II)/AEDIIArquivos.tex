\documentclass[a4paper, 12pt]{article}
\usepackage{tabto}    
\newcommand\mytab{\tab \hspace{-5cm}}
\usepackage[utf8]{inputenc}
\usepackage[normalem]{ulem}
\usepackage[dvipsnames]{xcolor}
\usepackage{amssymb}
\usepackage{algpseudocode}
\usepackage{amsmath}
\usepackage{mathtools}


 
\begin{document}
\title{AEDII Arquivos}
\author{•}
\date{}
\maketitle

\section{Pré-Árvore B}

\textcolor{white}{               }

\textcolor{blue}{Estrutura interna de arquivos de dados}\\

\rule{12cm}{0.4pt}\\

$\rightarrow$ Armazenamento não volátil também conhecidos como memória secundária, são aqueles que em que o arquivo gravado não se perde quando "desligado". Alguns exemplos de memória secundária são: HD's externo, DVD, Pen-drives entre outros. São de capacidade maior, porém lentas e baratas.\\

\rule{12cm}{0.4pt}\\

$\rightarrow$ Armazenamento volátil são aqueles que em que o que nele está se perde quando o computador é desligado. São os registradores, memória cache, e memória primária. Possuem capacidade menor, são rápidas e caras.\\

\rule{12cm}{0.4pt}\\

$\rightarrow$ Seeking: posiciona a cabeça de L/E no local desejado; o conteúdo de todo um cilindro poder ser lido com apenas 1 seeking; movimento mais lendo da operação de leitura/escrita; deve ser reduzido ao mínimo.\\

\rule{12cm}{0.4pt}\\

$\rightarrow$ O acesso do seek é direto (aleatório ou randômico), pois não há a necessidade de ler todos os dados em sequência. - Por outro lado, fitas precisam de um acesso sequencial.\\

\rule{12cm}{0.4pt}\\

$\rightarrow$ Paginação: capacidade de processar parte dos dados que estavam em memória secundária na memória principal (moldura de memória). Subconjunto do disco em memória principal.\\

\rule{12cm}{0.4pt}\\

\color{red}
$\rightarrow$ Memória virtual\\
\begin{itemize}
\item paginação
\item mapeamento de endereços
\item reposição de páginas
\begin{itemize}
\item LRU - Menos Recentemente Utilizada
\item LFU - Menos Frequentemente Utilizada
\item FIFO - Primeiro a Entrar Primeiro a Sair\\
\end{itemize}
\end{itemize}

\rule{12cm}{0.4pt}\\

\color{black}
$\rightarrow$ Organização de arquivos na memória secundária

\begin{itemize}
\item Bloco: unidade de transferência de dados entre memória principal e a memória secundária
\begin{itemize}
\item um bloco é formado por uma ou

\item É comum considerar que em cada bloco pode haver vários registros\\
Se R (tamanho fixo do registro, para
simplificar) e B (tamanho do bloco) e R $\leq$ B:\\
fator de blocagem fb = floor(B/R) = número de registros inteiros que cabem em um bloco
mais páginas
\end{itemize}

\item Organização espalhada: os blocos são
totalmente preenchidos; se um registro não
cabe inteiramente na parte vazia do bloco,
coloca o que couber e um ponteiro para o
próximo bloco

\item Organização não espalhada: registros não podem ser divididos. Cada bloco pode conter até fb registros.

\item Alocação de Blocos:

\fcolorbox{green}{white}{Blocos alocados sequencialmente}

\begin{itemize}
\item Leitura fácil (leitura sequencial é ótima, e na leitura aleatória depende da
facilidade de localização do deslocamento do registro dentro do arquivo)
\item Expansão complicada: se não houver espaço disponível até o próximo
arquivo tem que ser removido para outro local
\item Fragmentação externa (buracos entre os arquivos): maior ou menor dependendo da política de alocação - o disco pode ficar fragmentado, isto é, com vários trechos disponíveis intercortados por trechos utilizados
\end{itemize}

\begin{itemize}
\item Métodos de ajuste sequencial

\begin{itemize}
\item Primeiro ajuste: seleciona o primeiro trecho encontrado (a partir do início da lista) grande o suficiente - \textcolor{BlueViolet}{é o mais eficiente: balanço entre tempo de achar um bloco
de tamanho suficiente (retorna assim que achar o primeiro) e fragmentação
(não deixa sobrar sistematicamente o menor ou maior trecho)}

\item Próximo ajuste: seleciona o próximo trecho grande o suficiente (a partir do índice
“corrente”, ajustado após a última alocação) - \textcolor{BlueViolet}{similar ao primeiro ajuste, mas chega mais rápido ao fim do
heap}

\item Melhor ajuste: seleciona o menor trecho dentre os trechos grandes o suficiente - \textcolor{BlueViolet}{pode ser o pior: gasta tempo analisando tudo e, a menos que o ajuste seja perfeito, deixa sobrar normalmente trechos pequenos que não
podem ser reutilizados}

\item Pior ajuste: seleciona o maior trecho de todos - \textcolor{BlueViolet}{tenta evitar esse desperdício, deixando sobrar trechos maiores que podem ser ainda utilizados, e assim posterga a criação de blocos
pequenos}
\end{itemize}

\end{itemize}

\fcolorbox{green}{white}{Blocos alocados sequencialmente ordenado}

\begin{itemize}
\item Leitura ordenada eficiente (sequencial) – O(b): O próximo registro pode estar no mesmo bloco
\item Mínimo / Máximo estão no cabeçalho do arquivo –
O(1): Isto podemos fazer para todos os tipos de alocação
\item Busca: dá para usar busca binária (baseada nos blocos!)
\item Inserção: cara! O(b)
\begin{itemize}
\item Tem que achar a posição certa : O(lg b)
\item Tem que abrir espaço para o registro (deslocar
todos os\\ registros com chave maior para frente) :
O(b)
\end{itemize}

\item Exclusão: cara pelos mesmos motivos! - O(b)
\item Modificação: busca + atualização
\end{itemize}

\fcolorbox{green}{white}{\textcolor{red}{Blocos alocados por lista ligada}}

\fcolorbox{green}{white}{\textcolor{red}{Blocos com alocação indexada}}
\color{red}
\begin{itemize}
\item Índices primários
\item Índice de clustering
\item Índices secundários
\item Organização indexada multiníveis
\end{itemize}

\color{black}
\end{itemize}


\section{Árvore B}

\textcolor{white}{               }

$\rightarrow$ Com inspiração das árvores binárias de busca e com o dinamismo dos índices multiníveis cria-se a Árvore B.\\

\rule{12cm}{0.4pt}\\

$\rightarrow$  \fcolorbox{black}{white}{Definição: } 
\begin{itemize}
\item \fcolorbox{black}{white}{1. } Cada nó $x$ contém os seguintes campos

\begin{itemize}
\item $n[x]$, o número de chaves atualmente armazenadas no nó $x$;
\item as $n[x]$ chaves, armazenadas em ordem não decrescente, de modo que $key_{1}[x] \leq key_{2}[x] \leq ... \leq key_{n}[x]$;
\item $leaf[x]$, um valor booleano indicando se $x$ é um folha ($true$) ou um nó interno ($false$);
\item se $x$ é um nó interno, $x$ contém $n[x] + 1$ ponteiros $c_{1}[x], c_{2}[x], ... c_{n[x]+1}[x]$ para seus filhos.
\end{itemize}


\item \fcolorbox{black}{white}{2. } As chaves  $key_{i}[x]$ separam as faixas de valores armazenado em cada subárvore: denotando por $key_{i}$ uma chave qualquer armazenada na subárvore com nó $c_{i}[x]$, tem-se:

\begin{center}
$k_{1} \leq key_{1}[x] \leq k_{2} \leq key_{2}[x] \leq ... \leq key_{n[x]}[x] \leq k_{n[x]+1}$
\end{center}



\item \fcolorbox{black}{white}{3. } Todas as folhas aparecem no mesmo nível, que é a altura da árvore, $h$.

\item \fcolorbox{black}{white}{4. } Há um limite inferior e superior no número de chaves que um nó pode conter, expressos em termos de um inteiro fixo ${t \geq 2}$ chamado o \textit{grau mínimo} (ou $ordem$) da árvore.

\begin{itemize}
\item Todo nó que não seja a raiz deve conter pelo menos $t - 1$ chaves.
\item Todo nó interno que não seja a raiz deve conter pelo menos $t$ filhos.
\item Todo nó deve conter no máximo $2t - 1$ chaves (e portanto todo nó interno deve ter no máximo $2t$ filhos). Dizemos que um nó está $cheio$ se ele contiver exatamente $2t - 1$ chaves.
\end{itemize}



\item \fcolorbox{black}{white}{Altura. } Teorema: Para toda árvore B de grau mínimo $t \geq 2$ contendo $n$ chaves, sua altura $h$ máxima será:\\

\begin{center}
$h \leq \log_{t}\frac{n+1}{2}$
\end{center}


\end{itemize}

\rule{12cm}{0.4pt}\\

$\rightarrow$ Criação de uma árvore vazia\\

\begin{algorithmic}[1]
\State {BTreeCreate(T)}
\State {$i \gets AllocateNode()$}
\State {$leaf[x] \gets true$}
\State {$n[x] \gets 0$}
\State {$DiskWrite(x)$}
\State {$root[T] \gets x$}
\end{algorithmic}

\rule{12cm}{0.4pt}\\

$\rightarrow$ \textit{B-Tree-Search} toma como entrada um ponteiro para o nó de raiz x de uma subárvore e uma chave $k$ que deve ser
procurada nessa subárvore. Se $k$ está na B-árvore, \textit{B-Tree-Search} retorna o par ordenado $(y, i)$, que consiste em um nó $y$ e um índice $i$ tal que $key_{i}[y] = k$. Caso contrário, o procedimento retorna $NIL$. Assim, a chamada de nível superior é da forma \textit{B-Tree-Search}(root[T], k).\\

\begin{algorithmic}[1]
\State {BTreeSearch(x, k)}
\State {$i = 1$}
\While {$ i \leq n[x]$ e $k$\textgreater $key_{i}[x]$}
	\State {i = i + 1}
\EndWhile
\If {$i\leq n[x]$ e $k = key_{i}[x]$}
	\State \Return (x, i)
\ElsIf{$leaf[x]$}
	\State \Return NIL
\Else
       \State {$DiskRead(c_{i}[x])$}
\EndIf

\State \Return {$BTreeSearch(x.c_{i}, k)$}
\end{algorithmic}

\rule{12cm}{0.4pt}\\

$\rightarrow$ Inserção ocorrem sempre nas folhas

\begin{itemize}
\item Caso o nó em que a folha será inserida estiver cheia , alguma das chaves deverá ser promovida e esse nó se subdividirá.
\item Durante a busca da localização de inserção do nó, no caminho da raiz até uma folha, se achar um nó filho (a ser seguido) cheio, já o subdivide.
\item Vamos assumir que o nó atual (pai do nó cheio) não é cheio, a menos da raiz que deve ser tratada separadamente
\end{itemize}



\rule{12cm}{0.4pt}\\

$\rightarrow$ Divisão de um nó na árvore: \textit{BTreeSplitChild(x, i, y):} tem como entrada um nó interno \textit{x não cheio}, um índice $i$ e um nó $y$ tal que $y = c_{i}[x]$ é um filho $cheio$ de $x$. O procedimento divide $y$ em 2 e ajusta $x$ de forma que este terá um filho adicional.\\

\begin{algorithmic}[1]
\State {BTreeSplitChild(x, i, y)}
\State {$z \gets AllocateNode()$}
\State {$leaf[z] \gets leaf[y]$}
\State {$n[z] \gets t-1$}
\For {$j \gets 1$ $\textbf{to}$ $t-1$}
	\State {$key_{j}[z] \gets key_{j+t}[y]$}
\EndFor
\If {not $leaf[y]$}
	\For {$j \gets 1$ $\textbf{to}$ $t$}
		\State {$c_{j}[z] \gets c_{j+t}[y]$}
	\EndFor
\EndIf
\State {$n[y] \gets t-1$}
\For {$j \gets n[x]$ $\textbf{downto}$ $i$}
	\State {$key_{j+1}[x] \gets key_{j}[x]$}
\EndFor
\State {$key_{i}[x] \gets key_{t}[y]$}
\State {$n[x] \gets n[x]+1$}
\State {$DiskWrite(y)$}
\State {$DiskWrite(z)$}
\State {$DiskWrite(x)$}
\end{algorithmic}
\textcolor{white}{               }\\
$\rightarrow \rightarrow \rightarrow$ Linhas 2-10 Aloca e inicializa um nó para ser filho da chave que será promovida.\\
$\rightarrow \rightarrow \rightarrow$ Linha 13 Ajusta o nó que foi dividido.\\
$\rightarrow \rightarrow \rightarrow$ Linhas 14-18 Ajusta o nó em que foi parar o nó que saiu do nó dividido.

\rule{12cm}{0.4pt}\\

$\rightarrow$ Iserção de uma chave na árvore com raiz T:\\

\begin{algorithmic}[1]
\State {\textbf{BTreeInsert(T, k)}}
\State {$r \gets root[T]$}
\If {$n[r] = 2t - 1$}
	\State {$s \gets AllocateNode()$}
	\State{$root[T] \gets s$}
	\State{$leaf[s] \gets false$}
	\State{$n[s] \gets 0$}
	\State{$c_{1} \gets r$}
	\State{$BTreeSplitChild(s, 1, r)$}
	\State{$BTreeInsertNonFull(s, k)$}
\Else
	\State{$BTreeInsertNonFull(r, k)$}
\EndIf
\end{algorithmic}
\textcolor{white}{               }\\

$\rightarrow$ Iserção de uma chave em uma subárvore cuja raiz $x$ não está cheia:\\

\begin{algorithmic}[1]
\State {\textbf{BTreeInsertNonFull(x, k)}}
\State {$i \gets n[x]$}
\If {$leaf[x]$}
	\While {$i \geq 1$ and $k \textless key_{i}[x]$}
		\State {$key_{i+1}[x] \gets key_{i}[x]$}
		\State {$i \gets i-1$}
	\EndWhile
	\State {$key_{i+1}[x] \gets k$}
	\State {$n[x] \gets n[x] + 1$}
	\State{$DiskWrite(x)$}
\Else
	\While {$i \geq 1$ and $k \textless key_{i}[x]$}
		\State {$i \gets i-1$}
	\EndWhile
	\State {$i \gets i+1$}
	\State{$DiskRead(c+{i}[x])$}
	\If {$n[c_{i}[x]] = 2t -1$}
		\State {$BTreeSplitChild(x, i, c_{i}[x]$}
		\If {$k \textgreater key_{i}[x]$}
			\State {$i \gets i+1$}
		\EndIf
	\EndIf
	\State {$BTreeInsertNonFull(c_{i}[x], k)$}
\EndIf
\end{algorithmic}

\rule{12cm}{0.4pt}\\

$\rightarrow$ Remoção em árvores B - BTreeDelete(x, k): remoção da chave $k$ da subárvore com raiz $x$.

\begin{itemize}

\item \fcolorbox{black}{white}{2. } - Se a chave $k$ está no nó $x$ e $x$ é um nó interno, faça:

\begin{itemize}
\item a) Se o filho $y$ que pecede $k$ no nó $x$ tem pelo menos $t$ chaves, então encontre o predecessor de $k'$ de $k$ na subárvore com raiz $y$. Delete recursivamente $k'$, e substitua $k$ por $k'$ em $x$.

\item b) Simetricamente, se o filho $z$ imediatamente após $k$ no nó $x$ tem pelo menos $t$ chaves, então encontre o sucessor $k'$ de $k$ na subárvore com raiz $z$. delete recursivamente $k'$, e substitua $k$ por $k'$ em $x$.

\item c) Caso contrário, se ambos $y$ e $z$ possuem apenas $t-1$ chaves, faça a junção de $k$ e todas as chaves de $z$ em $y$, de forma que $x$ perde tanto a chave $k$ como o ponteiro para $z$, e $y$ agora contém $2t-1$ chaves. Então, libere $z$ e delete recursivamente $k$ e $y$.
\end{itemize}

\item \fcolorbox{black}{white}{3. } - Se a chave $k$ não está presente no nó interno $x$, determine a raiz $c_{i}[x]$ da subárvore apropriada que deve conter $k$ (se $k$ estiver presente na árvore). Se $c_{i}[x]$ tem apenas $t-1$ chaves, execute o passo 3a ou 3b conforme necessário para garantir que o algoritmo desça para um nó contendo pelo menos $t$ chaves. Então, continue no filho apropriado de $x$.

\begin{itemize}
\item a) Se $c_{i}[x]$ contém apenas $t-1$ chaves mas tem um irmão imediato com pelo menos $t$ chaves, dê para $c_{i}[x]$ uma chave extra movendo uma chave de $x$ para $c_{i}[x]$, movendo uma chave do irmão imediato de $c_{i}[x]$ à esquerda ou à direita, e movendo o ponteiro do filho apropriado do irmão para o nó $c_{i}[x]$.

\item b) Se $c_{i}[x]$ e ambos os irmãos imediatos de $c_{i}[x]$ contêm $t-1$ chaves, faça a junção de $c_{i}[x]$ com um de seus irmãos. Isso implicará em mover uma chave de $x$ para o novo nó fundido (que se tornará a chave mediana para aquele nó).
\end{itemize}

\item \fcolorbox{black}{white}{1. } - Se a chave $k$ está no nó $x$ e $x$ é uma folha, exclua a chave $k$ de $x$. - (pelos procedimentos anteriores, já sabemos que o nó $x$ tem pelo menos $t$ chaves).

\item \fcolorbox{black}{white}{*** } Quando um dos nós for a raiz
\begin{itemize}
\item Ela pode ter menos do que $t-1$ chaves
\item Se ficar com zero chaves precisa desalocar o bloco e atualizar quem é a nova raiz.
\item Precisa de uma camada extra sobre a chamada da deleção:

\begin{algorithmic}[1]
\State {\textbf{BTreeDeleteFromRoot(T, k)}}
\State {$r \gets raiz[T]$}
\If {$n[r] = 0$}
	\State \Return
\Else 
	\State {$BTreeDelete(r,k)$}
\EndIf
\If {$n[r] = 0$ $\textbf{and not}$ $leaf[r]$}
	\State {$raiz[T] \gets c1[r]$}
	\State {$desaloca(r)$}
\EndIf
\end{algorithmic}


\end{itemize}

\end{itemize}



\subsection{Hashing (Espalhamento): Tenenbaum cap 7.4} 

\begin{itemize}
\item Considere uma conjunto de chaves $k$ e seja $I$ conjunto de índices.\\
$\rightarrow$ $I = \{0, 1, 2, ... , I-1\}$\\


\begin{table}[h]
\centering
\caption{Tabela com índices}
\begin{tabular}{|c|}
 \hline 
 0\\ 
 \hline
 1\\
 \hline
 2\\
 \hline
 .\\ 
 \hline
 .\\ 
 \hline
 .\\ 
 \hline
 $I-1$\\ 
 \hline
\end{tabular}
\end{table}

\item Uma função hash, denotada por $h(k)$\\
$h: k \rightarrow I$ para cada chave associa-se um único índice. A idéia é que o custo de uma busca seja $O(1)$.
\begin{itemize}
\item Ex: $h(k) = k$ $\%$ $T$, onde $T$ é o tamanho do conjunto de índices.

\begin{equation}
  {Colisao} =
    \begin{cases}
      {h(53)} = 3\\
      {h(13)} = 3\\
    \end{cases}       
\end{equation}

*colisão é o fato de duas chaves distintas $k_{1}$ e $k_{2}$ serem associadas ao mesmo índice.
\end{itemize}

\item Tratamento de Colisões\\
$\rightarrow$ Endereçamento aberto: consiste em uma função chamada de $rehash - rh(k)$\\
\\
\color{white}
$\rightarrow$
\color{black}
$\rightarrow$ $rh: I \rightarrow I$\\
\color{white}
$\rightarrow$
\color{black}
$\rightarrow$ Essa função é responsável por ``realocar" a chave na qual houve uma colisão.\\
\color{white}
$\rightarrow$
\color{black}
*há um limite no tamanho da lista e o tempo de busca.\\
\\
$\rightarrow$ Dadas funções $h(k)$ e $rh(i)$
\begin{algorithmic}[1]
\State {\textbf{bool search (int k)}}
\State {$int$ $p = h(k)$}
\While {$(T[p]$ \textbf{is not} $k)$ \textbf{and} $(T[p]$ \textbf{is not} $-1))$}
		\State {$p = r//h(k)$}
	\EndWhile
\If {$T[p] == -1$}
	\State \Return false
\Else 
	\State \Return true
\EndIf
\end{algorithmic}
\color{white}
$\rightarrow$
\color{black}
o laço entrará num loop infinito caso a tabela esteja totalmente cheia. Uma alternativa é definir como cheia a $tabela-1$ para garantir que a busca termine.\\

\begin{algorithmic}[1]
\State {\textbf{bool insert (int k)}}
\State {$int$ $p = h(k)$}
\While {$(T[p]$ \textbf{is not} $k)$ \textbf{and} $(T[p]$ \textbf{is not} $-1))$}
		\State {$p = rh(p) $}
	\EndWhile
\If {$T[p] == -1$}
	\State {$T[p] = k$}
	\State \Return true
\Else 
	\State \Return false
\EndIf
\end{algorithmic}

\color{white}
$\rightarrow$
\color{black}

\begin{algorithmic}[1]
\State {\textbf{bool remove (int k)}}
\State {$int$ $p = h(k)$}
\While {$(T[p]$ \textbf{is not} $k)$ \textbf{and} $(T[p]$ \textbf{is not} $-1))$}
		\State {$p = rh(p) $}
	\EndWhile
\If {$T[p] == k$}
	\State {$T[p] =$ $?$ \textcolor{gray}{//o que fazer?}}
\EndIf
\end{algorithmic}

\color{white}
$\rightarrow$
\color{black}
Atualizar as funções de busca e inserção considerando o marcador de removido.\\
\color{white}
$\rightarrow \rightarrow$
\color{black}
$\rightarrow$ a função de busca não muda\\
\color{white}
$\rightarrow \rightarrow$
\color{black}
$\rightarrow$ a função de inserção precisa considerar o marcador de removido.\\

$\downarrow$ insert modificado.
\begin{algorithmic}[1]
\State {\textbf{bool insert (int k)}}
\State {$int$ $p = h(k)$}
\State {$int$ $temp = -1$}
\While {$(T[p]$ \textbf{is not} $k)$ \textbf{and} $(T[p]$ \textbf{is not} $-1))$}
		\If{$T[p] == -2$}
			\State{$temp = p$}
		\EndIf
		\State {$p = rh(p)$}
	\EndWhile
\If {$T[p] == k$}
	\State \Return false
\Else
	\If {$temp$ \textbf{is not} $-1$}
		\State {$T[temp] = k$}
	\Else
		\State {$T[p] = k$}
		\State \Return true
	\EndIf	
\EndIf
\end{algorithmic}

\begin{itemize}
\item  \fcolorbox{white}{pink}{Eficiência dos métodos de $rehash$}\\
\\
$\rightarrow$ Uma medida de eficiencia é contar o número de posições examinadas(probabilidade) antes de encontrar uma chave.\\
$\rightarrow$ O número médio (valores altos de T) de consultas para $\frac{2T - n+1}{2T - 2n+2}$ onde $n$ é o número de chaves presentes na tabela.\\
$\rightarrow$ Define-se por \textit{lead factor} $(lf)\frac{n}{T}$ (fração da tabela que está preenchida)\\
$\rightarrow$ Inicialmente todas as posições da tabela tem a mesma probabilidade de serem preenchidas. A medida que a tabela vai enchendo, algumas posições tem um aumento nessa probabilidade.\\
$\rightarrow$ Uma forma de eliminar a \fcolorbox{black}{white}{aglomeração primária} é fazer com que a função $rh()$ dependa de dois argumentos $rh(i,j)$ onde $i$ é o número de vezes que a função $rh()$ for aplicada.\\ $rh(i, j) = (i+j)$ $\%$ $T$.\\
\\
$rh(i, 1) = (i+1)$ $\%$ $T$\\
$rh(i, 2) = (i+2)$ $\%$ $T$\\
$rh(i, 3) = (i+3)$ $\%$ $T$\\

$\rightarrow$ Uma variação dessa técnica é usar uma permutação aleatória de $0$ até $T-1$.\\
\begin{center}
$(p_{0}, p_{1}, p_{2}, ... ,p_{t+1})$ e define o $rh(j)$ como $(h(r) + pj)$ $\%$ $T$
\end{center}
$\rightarrow$ Uma terceira abordagem é $rh(j) = (h(r) + j^2)$ $\%$ $T$.\\
$\rightarrow$ Uma quarta opção é $rh(i, k) = (i + h_{key})$ $\%$ $T$.\\
\color{white}
$\rightarrow \rightarrow \rightarrow \rightarrow \rightarrow \rightarrow \rightarrow \rightarrow \rightarrow \rightarrow \rightarrow \rightarrow \rightarrow \rightarrow \rightarrow \rightarrow \rightarrow$
\color{black}
$\hookrightarrow$ $h_{key} = (1 + h(k))$ $\%$ $T$.\\ 

\item \fcolorbox{white}{pink}{Aglomeração Secundária}\\
$\rightarrow$ chaves com o mesmo valor de $hash$, isto é, $k_{1}$ $\neq^*$ $k_{2}$ e $h(k_{1})$ = $h(k_{2})$ tendem a seguir o mesmo $rehash$ até serem inseridas.\\
$\rightarrow$ uma forma de eliminar a aglomeração secundária é chamada de \emph{duplo hash} e usa duas funções $hash$ $h_{1}$ e $h_{2}$ usada na fase inicial, isto é, para determinar a posição da chave na tabela. Se a posição estiver ocupada o $rehash$ fica dessa forma:
\begin{center}
$rh(i, k) = (1 + h_{2}(k))$ $\%$ $T$
\end{center}
\end{itemize}

\end{itemize}
\pagebreak

\subsection{Tabela hash ordenada}

A idéia geral é manter as chaves que colidirem na mesma posição em ordem decrescente.

\begin{algorithmic}[1]
\State {\textbf{int insert (int key)}}
\State {$int$ $i, $ $j, $ $tempKey, $ $nKey$}
\State {$bool$ $first$}
\State {$i = h(key)$}
\State {$nKey = key$}
\State {$first = true$}
\While {$Table[i] > nKey$}
	\State {$i = rh(i)$}
\EndWhile
\State {$int$ $tk$ $= Table[i]$}
\While {$( (tk$ \textbf{is not} $-1)$ \textbf{and} $(tk$ \textbf{is not} $nKey)$}
	\State {$tempKey = Table[i]$}
	\color{BlueViolet}
	\If {$tempKey$ $\textless$ $nKey$}
		\color{black}
		\State {$Table[i] = nKey$}
		\State {$nKey = tempKey$}
		\If{$first$}
			\State {$j = i$}
			\State {$first = false$}
		\EndIf
	\color{BlueViolet}
	\EndIf
	\color{black}
		\State {$i = rh(i)$}
		\State {$tk = Table[i]$}
	
\EndWhile
\If {$tk == -1$}
	\State {$Table[i] = nKey$}
\EndIf
\If {$first$}
	\State \Return {$i$}
\Else
	\State \Return {$j$}
\EndIf

\end{algorithmic}

O que permite que as chaves fiquem ordenadas e que a busca seja bem-sucedida é o $if$ $(13-20)$. Com ele as chaves só serão movidas se o elemento a ser inserido for maior que a chave em que está localizado o seu hash.
\pagebreak

\subsection{Novos métodos para a função \textit{hash}}

A função hash que estudamos até agora foi a do método da divisão:

\begin{center}
$h(chave) =$ $chave$ $\%$ $T$, $T$ é o tamanho da Tabela Hash.
\end{center}

Observação: chave é um número inteiro(int)\\

O que torna a função eficiente? R: Minimizar o número de colisões.\\
\\
\\
Para os novos métodos e pensando na eficiência deles considere os seguintes critérios:

\begin{itemize}
\item Critério 1) Levar em consideração toda a informação contida na chave (não só o último número que era considerado no método da colisão)
\item Critério 2) Sensibilidade a permutação (Os números 53 e 33 cairiam no mesmo índice da tabela pelo hash, mesmo sendo números diferentes, para os métodos seguintes não é isso que se deseja).
\item Critério 3) Trabalhar com chaves alfa-numéricas.
\end{itemize}
\textcolor{white}{$\rightarrow$}
\\
\textbf{Métodos:}
\begin{itemize}
\item \fcolorbox{black}{white}{1 - } O método da divisão consegue melhorar resultados quando 

TableSize é primo.

\item \fcolorbox{black}{white}{2 - } Método multiplicativo $h(chave)$ é definido como 
\begin{center}
$floor(m * frac(c * chave))$,
\end{center}
onde $c$ é um número real entre 0 e 1, e $m$ é o tamanho da Tabela.
Valores de $c$ com boas propriedades:
\begin{center}
$c_{1} = 0,6180339887$,\\
$c_{2} = 0,3819660113$.\\
\end{center}
Este método usa toda a informação da chave e é sensível a permutações.
 
\pagebreak
\item \fcolorbox{black}{white}{3 - } Método do quadrado médio (para TableSize = potência de dez)\\
O valor da chave é elevada ao quadrado e alguns dígitos do ``meio" são usados como índice.\\
\textcolor{white}{$\rightarrow$} Exemplo:

\textcolor{white}{$\rightarrow \rightarrow \rightarrow$}$T = 1000$ (índices de 0 a 999)\\
\textcolor{white}{$\rightarrow \rightarrow \rightarrow$}$chave = 245$
\begin{center}
$245^2$ = 60025 $\rightarrow$ $6\fcolorbox{black}{pink}{002}5$ $\rightarrow$ $h(245) = 002$.
\end{center}
Este método usa toda a informação da chave e é sensível a permutações.

\item \fcolorbox{black}{white}{4 - } Método da dobra\\
Suponha uma representação binária de uma chave que segue: 
\begin{center}
010111001010110.
\end{center}
Dividindo este número em três partes:
\begin{center}
$\underbracket{01011}_{}$ $\underbracket{10010}_{}$ $\underbracket{10110}_{}$
\end{center}
E empilando para fazer operações com XOR:
\begin{center}
01011\\
10010\\
\underline{10110}\\
01111\\
\rotatebox[origin=c]{180}{$\Lsh$} $15_{10}$
\end{center}
Índice será 15.\\
Este método usa toda a informação da chave e é sensível a permutações.

\item \fcolorbox{black}{white}{5 - } Chaves alfa-numéricas\\
Strings; char[50];\\
\\
chave = ``hello"\\
Podemos interpretar uma string na base 26(alfabeto)

$\underbrace{h}_{8*26^4+}$ $\underbrace{e}_{5*26^3+}$ $\underbrace{l}_{12*26^2+}$ $\underbrace{l}_{12*26^1+}$ $\underbrace{o}_{15*26^0}$ = 3.752.127\\

Achando o resultado pode-se usar qualquer um dos métodos anteriores, pois com essa etapa já é o suficiente para garantir a eficiência.
Este método contempla os três critérios.
\end{itemize}

\begin{algorithmic}[1]
\State {\textbf{int Quadrados (int c)}}
\State {$int$ $cs = $ $c*c$}
\If {$((cs$ $\%$ $1000)$ $== cs)$}
	\State \Return {$cs$}
\EndIf
\State {$int$ $nd = $ $0$}
\State {$int$ $cstemp = $ $cs$}
\While {$cstemp > 0$}
	\State {$nd++$}
	\State {$cstemp$ $/= 10$}
\EndWhile
\State {$int$ $dt = $ $\log(1000)$} \textcolor{gray}{//3}
\State {$int$ $r = $ $nd - dt$} \textcolor{gray}{//saber quantos dígitos retirar}
\State {$r$ $/=$ $2$} \textcolor{gray}{//funciona para par e ímpar}
\State {$cstemp$ $=$ $cs$}
\For {$int$ $i = 0; i $ $\textless$ $r;$ $i++$}
	\State {$cstemp$ $/=$ $10$}
\EndFor
\State \Return {$cstemp$ $\%$ $1000$} \textcolor{gray}{//Pega os três dígitos do final (60025 - 5 já saiu)}

\end{algorithmic}

\section{Notas}
\textcolor{white}{.}\\
$\rightarrow$ O que está de \textcolor{red}{vermelho com asterisco*} precisa ver se está correto.\\
$\rightarrow$ O que está somente de \textcolor{red}{vermelho} é para incluir no arquivo.

\end{document}

